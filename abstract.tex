%----------------------------------------------------------------------------------------
%	ABSTRACT PAGE
%----------------------------------------------------------------------------------------

\addtotoc{Abstract} % Add the "Abstract" page entry to the Contents
\addtocontents{toc}{}
\btypeout{Abstract Page}
\thispagestyle{plain}
\null\vfil
\begin{center}
	\setlength{\parskip}{0pt}
	{\huge{\textit{Abstract}} \par}
	\bigskip
	{\Large\bf \ttitle \par} % Thesis title
	\medskip
	{\normalsize by \authornames \par} % Author name
\end{center}

Data mining enables a better understanding of human and natural processes by analyzing massive amounts of data with machine learning algorithms. Stream mining is a process that allows us to discover knowledge in data when it comes in the form of a continuous stream. MOA, initials for Massive Online Analysis, is an open source data stream mining framework, developed at the University of Waikato, New Zealand. One of the available features in MOA is the use of filters, which can process streaming data before or after being fed to other subsystems, such as machine learning algorithms.

Although data science has brought us many benefits, because the data being analyzed is often personal and sensitive, we face the threat of losing our privacy. Statistical Disclosure Control (SDC) deals with controlling that information about specific individuals is not extracted from released datasets, whilst maintaining the statistical significance of the masked data. By applying SDC techniques to data, disclosure is prevented, thus effectively protecting the privacy of the data owners.

Four MOA \textit{privacy-preserving filters} have been developed in this project, which implement the following SDC methods: \textbf{noise addition}, \textbf{microaggregation}, data \textbf{rank swapping} and \textbf{differentially private microaggregation}. Each of the algorithms has been adapted from well-known solutions in order to enable their utilization for stream processing tasks. Finally, the filters have been benchmarked to assess their quality in terms of two important SDC measurments: disclosure risk and information loss.

\clearpage % Start a new page

\addtotoc{Abstract (Catalan)} % Add the "Abstract" page entry to the Contents
\addtocontents{toc}{}
\btypeout{Abstract Page}
\thispagestyle{plain}
\null\vfil
\begin{center}
	\setlength{\parskip}{0pt}
	{\huge{\textit{Resum}} \par}
	\bigskip
	{\Large\bf \ttitle \par} % Thesis title
	\medskip
	{\normalsize per \authornames \par} % Author name
\end{center}

La mineria de dades ens ajuda a entendre millor els processos antropogènics i naturals, analitzant quantitats massives de dades, mitjançant algorismes d'aprenentatge automàtic. La mineria de \textit{fluxos} de dades és un paradigma d'anàlisi que ens permet extreure coneixement de les dades quan són rebudes en forma d'un flux continu. El paquet de programari MOA, de l'anglès \textit{Massive Online Analysis}, és un entorn de mineria de fluxos de codi obert, desenvolupat a la Universitat de Waikato, a Nova Zelanda. Una de les funcionalitats de MOA és la possibilitat d'utilitzar filtres, els quals processen les dades en flux abans o després de ser redirigides cap a altres sub-sistemes, com ara algorismes d'aprenentatge automàtic.

Tot i que la mineria de dades ens aporta molts beneficis, les dades que s'analitzen són, sovint, personals i sensibles. Ens trobem, doncs, davant d'un escenari en el que la nostra privacitat està en perill. L'SDC, de l'anglès \textit{Statistical Disclosure Control}, és un camp que estudia mecanismes per controlar que la informació d'un individu específic no sigui extreta dels conjunts de dades publicades, alhora que s'intenta preservar la utilitat estadística de les dades emmascarades. Aplicant tècniques d'SDC, s'impedeix la re-identificació dels individus, protegint, per tant, la privacitat dels mateixos.

En aquest projecte s'han desenvolupat quatre \textit{filtres de preservació de la privacitat} per l'entorn MOA, que implementen els següents mètodes d'SDC: \textbf{addició de soroll}, \textbf{microagregació}, \textbf{intercanvi de rangs} i \textbf{microagregació de privacitat diferencial}. Cadascun dels algorismes ha estat adaptat d'algun mètode ja conegut, en ús, per habilitar la seva utilització per a tasques de processament de fluxos. Finalment, tots quatre filtres han estat avaluats respecte de dues mesures molt importants en l'àmbit de l'SDC: el risc de revelació i la pèrdua d'informació.

\clearpage % Start a new page

\addtotoc{Abstract (Spanish)} % Add the "Abstract" page entry to the Contents
\addtocontents{toc}{\vspace{1em}}
\btypeout{Abstract Page}
\thispagestyle{plain}
\null\vfil
\begin{center}
	\setlength{\parskip}{0pt}
	{\huge{\textit{Resumen}} \par}
	\bigskip
	{\Large\bf \ttitle \par} % Thesis title
	\medskip
	{\normalsize por \authornames \par} % Author name
\end{center}

La minería de datos nos ayuda a entender mejor los procesos antropogénicos y naturales, analizando cantidades masivas de datos, mediante algoritmos de aprendizaje automático. La minería de \textit{flujos} datos es un paradigma de análisis que nos permite extraer conocimiento de los datos, cuando éstos son recibidos en forma de un flujo continuo. El paquete de software MOA, del inglés \textit{Massive Online Analysis}, es un entorno de minería de flujos de código abierto, desarrollado en la Universidad de Waikato, Nueva Zelanda. Una de las funcionalidades de MOA es la posibilidad de utilizar filtros, los cuales procesan los datos de los flujos antes o después de ser redirigidos hacia otros subsistemas, como los algoritmos de aprendizaje automático.

Aunque la minería de datos nos aporta muchos beneficios, los datos que se analizan son frecuentemente personales y sensibles. Nos encontramos, pues, ante un escenario en el que nuestra privacidad está en peligro. El campo de SDC, del inglés {Statistical Disclosure Control}, estudia los mecanismos para controlar que la información de un individuo específico no se extraiga de los conjuntos de datos publicados, a la vez que se intenta maximizar la utilidad estadística de los datos enmascarados. Aplicando técnicas de SDC, se impide la re-identificación de los individuos, protegiendo, por lo tanto, la privacidad de los mismos.

En este proyecto se han desarrollado cuatro \textit{filtros de preservación de la privacidad} para el entorno MOA, que implementan los siguientes métodos de SDC: \textbf{adición de ruido}, \textbf{microagregación}, \textbf{intercanvio de rangos} y \textbf{microagregación de privacidad diferencial}. Cada uno de los algoritmos ha sido adaptado de algun método ya conocido y en uso, para habilitar su utilización para tareas de procesamiento de flujos. Finalmente, los cuatro filtros se han evauado respecto dos medidas muy importantes en el ámbito del SDC: el riesgo de revelación y la pérdida de información.

\clearpage % Start a new page