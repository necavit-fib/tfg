\section{Future work}
\label{Conclusions:Future}

Different aspects of this project deserve to be considered for future enhancements, ranging from algorithmic details to benchmarking methods.

First and foremost, there is a particular non-functional requirement that could be emphasized in the future: documentation. The confection of a user manual for the MOA extension that the privacy filters represent would be a great complement to the toolsuite. In addition, the developed package should be released, as a binary distribution, to a central repository, in order to ease access not only to the source code, that is already public, but to ready for use bundles.

Concerning the algorithmic facet of the project, there is room for performance improvement, in terms of computational complexity and execution time. Advanced and more customized data structures could be used, as well as some other design approaches. Moreover, a complete code base refactoring should be carried out to adapt the SDC methods to the changes introduced by the latest MOA release, that was published while developing the filters for this project.

Finally, a more rigorous benchmark process could also be set up, running multiple executions to assess statistically valid results and applying the filters to real world dataset, for example, all of which could not be done due to the lack of time available.