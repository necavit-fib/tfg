\section{Achieved goals}
\label{Conclusions:Achieved}

The present report assesses the inception, development and benchmarking of the implementation of several Statistical Disclosure Control (SDC) methods, adapted to the MOA data stream mining framework. Concerning the actual goals of the project, we can say that it has ultimately been successful: the main requirements have been fulfilled and the resulting work has proven its performance and utility for the data science community.

First, a thorough theoretical background analysis had to be done in order to select the approaches best suited to the implementation of the filters on the MOA environment. Moreover, the state of the art of existing solutions was reviewed and some technology alternatives were considered before the actual development of the algorithms.

Four MOA privacy preserving filters have been developed, implementing the following SDC methods: \textit{noise addition}, \textit{microaggregation}, \textit{data rank swapping} and a microaggregation based \textit{differential privacy} mechanism. Each of the algorithms was adapted from well-known solutions, already in use in non-streaming data analysis settings, in order to enable their utilization in stream processing tasks. Special emphasis has been put in easing the filters customization, either by setting the appropriate parameterizations or by actually modifying parts of their behaviour exploiting the extensibility that the MOA framework offers us. Finally, all four filters have been benchmarked to assess their quality in terms of two important SDC measurments: \textit{disclosure risk} and \textit{information loss}. While some of the methods perform better than others, they all offer results that conform to their theoretical limits.

On the personal side, the development of this project has brought me a closer understanding of SDC, a field that was almost unknown to me, and has reassured the interest I have for two of the main concerns of the project: data science and the relation between technology and society, since privacy preservation is, indeed, the project's most important outcome. Beyond this, the possibility of working in an open source project and being able to contribute to its extension has proved to be really engaging.

Finally, I am certainly happy of having found a project in which all the skills and knowledge acquired throughout these years as an undergraduate student could be put in practice. Not only algorithmical theory and statistics-related concepts have been used, but also analytical rigour and good practices in terms of software architechture and development were necessary for the project's success.