\section{Legal framework}
\label{Practical::Legal}

One of the aspects to bear in mind when developing a technological project is the legal environment in which it is framed. To this respect, efforts are being carried out to develop legal frameworks to help protect people’s privacy, at many levels. One such example is the spanish LOPD\footnote{LOPD stands for L\textit{ey} O\textit{rgánica} de P\textit{rotección} de D\textit{atos}, a law that was approved by the spanish courts in 1999. It has been modified several times, being the law enforcement regulation approved in 2007.}, a law that aims, among other things, to define different data privacy levels and mandatory proceedings associated to each - no matter the medium used to transfer it or store it. The full text of the law can be consulted at the~\citet{BOE:LOPD}.

There are some pitfalls to these legislative efforts, though. Firstly, it is really hard to assess their accomplishment in the IT sector and, thus, it is sometimes a matter of confidence in the developer’s good practice. Another important drawback is that online services, such as social networks, can be accessed globally, but, on the other hand, their legislative framework is that of the country to which the backing company offering the service belongs to - jurisdiction definition in the Internet is still a matter of intense debate nowadays\footnote{Proof of this debate is the emergence of initiatives like the Internet \& Jurisdiction Project, which was launched in 2012 to address the tension between the cross-border nature of the Internet and the patchwork of national jurisdictions. To enable the digital coexistence of different norms in shared cross-border online spaces, it facilitates a neutral multi-stakeholder dialogue process, which brings together governments, civil society groups, major Internet platforms, technical operators and international organizations~\citep{web:InternetJurisdiction}.}.

We have not detected any kind of legal consequences or regulations bound to this project's development, besides intellectual property protection measures --- no personal data has yet been used to perform any benchmarking process nor to assess the quality of the developed methods: random data generators are being used instead (see~\cref{Chapter7Benchmarking}).

As we already stated before, concerning the code base of the project, we must implement all necessary copyright protection mechanisms. Because this is an \textit{open source} project, an internationally recognised software license is included in the public code repository, hosted at GitHub\footnote{The project is available at \url{https://github.com/necavit/moa-ppsm}.}. The chosen license is the MIT License, which has proven to be easy to understand, relatively widespread and quite permissive in terms of its commercial applicability.