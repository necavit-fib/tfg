\section{Privacy \& society}
\label{Practical::Privacy}

Privacy has become a hot topic in debates nowadays, concerning \textit{what} information is collected from individuals, \textit{who} owns it and with \textit{which} purposes. It is a matter of great importance and certainly worth to be examined carefully. Information technologies have brought us many benefits at many levels --- safer streets, cheaper communications, better health systems, more convenient shopping --- but many times at the high cost of losing our privacy. With the rapid adoption of the Internet and all sorts of digital telecommunications as the basis of our modern communication relationships, a vast capacity of interception, storage and analysis of such information exchanges has been reached. This potential has been used by companies in the private sector to, for example, analyze the population consuming profiles, target marketing campaigns more accurately and offer much more customized products and services. In order to apply these techniques and mechanisms, corporations collect private data from users, excusing that these same users accept privacy terms and conditions. It seems clear that data mining is highly related to privacy: knowledge discovery processes need data to work and, in most cases, sensitive personal data is at stake.

We have already outlined in~\sref{Theory:SDC} that the aim of SDC and this project in particular is to protect users privacy by avoiding information disclosure from released datasets and real time analysis processes that require sensitive data. The question, however, is: why do we \textit{need} to protect privacy? What urges us to preserve our right to privacy? It is not a simple and mere question; indeed, the answer is related to our understanding and interpretation of the term ``privacy'' itself. Therefore, we will review the definition of privacy and provide an argument that is the basis to justify privacy protection.

In the introductory chapter of the report (see~\sref{Introduction::Context::Privacy}) an introduction to the concept of privacy was given by literally reproducing a dictionary definition: ``\textit{Privacy is a concept that can be defined as the ability of an individual or group to seclude themselves, or information about themselves, and thereby express themselves selectively}''. We also saw that privacy is recognised as one of our most fundamental rights, as it is enshrined in the Universal Declaration of Human Rights. Going further on, \textit{privacy}, understood not only as the mechanism that allows us to keep our opinion and ideas private, but also the rest of our \textit{praxis}, enables us to develop a \textit{particular} personality, yet when we are within a social structure. Without the right to keep certain aspects of our life private, the \textit{individuation} process is compromised and many consequences of this individual diversity are endangered --- thought heterogeneity, for example, cultural heritage and, above all, individuals \textit{emancipation}, all because the individuation process does not happen in a context of complete freedom.

We must not forget that when organizations such as enterprises or governments acquire massive amounts of private information about particular individuals, a certain \textit{control} capacity on these individuals is gained too. This power, on the contrary of what ultimate defendants of data gathering hold, does not liberate people nor make them safer. The true consequence of such an increase in control power is that all equitable bonds between individuals and these organisms are torn apart: people become \textit{dominated} by social institutions, be them governments or any kind of structured association, and their freedom is, thus, canceled. There is no possible emancipation nor conviviality of people in a social context if the individuals-society relationships are domain based.

Finally, from a more pragmatic point of view, not only ethical concerns are addressed by protecting users privacy, but economical issues too. Industrial-scale information theft has a huge impact on enterprise economies, because of distrust and because disclosed sensitive data can be used to make profit of it. Identity theft, for example, was estimated to have a cost in the order of billions of dollars, back in 2005, as shown by~\citet{Romanosky:DisclosureLaws}.

\subsection{Impact of this project}
\label{Practical::Privacy:Impact}

The motivation of the project is now well-founded: privacy is a relevant concern for any data analysis related field, whether it is statistics, data mining or data stream processing. Of course, this project addresses just a small portion of the broader picture of privacy protection, but it is indeed another effort taken towards its effective achievement.

Together with good IT security practices, a reasonable usage of data and information and ackowledged consent from the data owners, the application of SDC techniques --- like the ones implemented by the privacy filters which conform the goal of this project --- enables the preservation of the inalienable right to privacy.

To provide further examples of the impact of the project, potential users of the MOA privacy filters are both companies and government statistical agencies, which handle vasts amounts of sensitive and personal data. Using SDC methods, they would be able to exploit the intrinsic knowledge of these data, while preserving privacy and protecting their users against disclosure attacks. Not only they could carry more interesting experiments, but they could also release this information, sharing it with third parties to promote collaboration with researchers and, last but not least, as an exercise of transparency.