Having reviewed the main concepts to which this project is related, we can now outline
its main purpose: \textbf{implement privacy preserving methods in a stream mining environment}.

\subsection{MOA}

\textbf{MOA}, initials for \textbf{M}assive \textbf{O}nline \textbf{A}nalysis, is an 
open source framework for data stream mining~\citep{web:MOA}, originally 
developed at the University of Waikato, New Zealand. It includes several machine learning 
algorithms\footnote{Algorithms used to perform the actual data mining analysis (the 
“machine learning \& visualization” step on \fref{fig:data-mining}) belong to the 
field of machine learning. In MOA, clustering, classification, regression, outlier 
detection and recommender systems are available.} to perform the analysis and tools 
to evaluate the quality of the results. It also deals with a problem known as 
\textit{concept drift}\footnote{It is said of statistical properties of a target variable 
being analyed, when they change over time in unforeseen ways.}. It is related to the well
known and commonly used Weka\footnote{Weka is a popular software package including 
classical data mining algorithms, this is, not stream mining. It is also developed at 
the University of Waikato.~\citep{web:Weka}} package, but it is built to perform at 
a greater scale for more demanding problems.

\subsubsection{MOA filters}

One of the available features in MOA is the use of \textit{filters}, which can process
streaming data before or after being fed to other systems or algorithms, such as learners
or file writers. However, few filters are currently shipped within the latest MOA distribution,
namely a filter to replace \textit{missing values}\footnote{In statistics, missing data, or missing values, 
occur when no data value is stored for the variable in an observation. Missing data are a 
common occurrence and can have a significant effect on the conclusions that can be drawn 
from the data.} and a filter that adds noise to data.

\subsubsection{MOA extensions}

When working with MOA, the environment consists of the core library, but \textit{extensions}
can be used to enhance the existing methods or to provide additional features, based on
the core tools that MOA already provides. A series of extensions have been developed and can
be found on MOA's website, at \url{http://moa.cms.waikato.ac.nz/moa-extensions}.

\subsection{The project in a nutshell}