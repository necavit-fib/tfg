% !TeX document-id = {78834bc9-5b2d-4839-9368-11a8c7d7974b}
% !TeX TXS-program:compile = txs:///pdflatex/[--shell-escape]

\documentclass[a4paper]{scrartcl}
\usepackage[utf8]{inputenc}
\usepackage[catalan]{babel}
\usepackage[T1]{fontenc}
\usepackage{microtype}
\usepackage{graphicx}
\usepackage[hidelinks]{hyperref}
\usepackage{color}
\usepackage{minted}
\usemintedstyle{friendly}
\usepackage{todonotes}
\usepackage[section]{placeins}
\usepackage{authblk}
\usepackage{titlesec}
\usepackage{ifthen}

\newcommand*{\appendixmore}{%
	\renewcommand*{\othersectionlevelsformat}[1]{%
		\ifthenelse{\equal{##1}{section}}{\appendixname~}{}%
		\csname the##1\endcsname\autodot\enskip}
	\renewcommand*{\sectionmarkformat}{%
		\appendixname~\thesection\autodot\enskip}
}


\title{Endevina Paraules (IV)}
\date{Juny 2014}
\author{Daniel Ariñez Soriano}
\author{David Martínez Rodríguez}
\author{Miquel Masriera Quevedo}
\author{Marcel Pujol Navarro\vspace{11cm}}
\affil{Arquitectura del Software\\Facultat d'Informàtica de Barcelona, UPC}

\renewcommand\Authands{ i }

\begin{document}
	
	\maketitle
	
	%\begin{center}
	% \includegraphics[width=3cm]{images/logo.png}
	%\end{center}
	
	\thispagestyle{empty}
	\newpage
	\cleardoublepage
	
	\section{Repositori de codi}
	El repositori de codi de la pràctica es troba a la següent URL: \url{https://github.com/necavit/AS}.
	El repositori és de lliure accés, de manera que no hauria de ser un problema per descarregar-se els
	continguts o per navegar-hi.
	El codi es troba comentat i en la seva última versió.
	A l'arrel del repositori hi figura el fitxer \texttt{README.md}, on es detallen alguns conceptes sobre
	l'estructura de fitxers del repositori, la configuració del projecte o la seva execució.
	
	\section{Canvis respecte la demostració}
	Hem arreglat alguns dels defectes que presentava l'aplicació a la demostració d'avui, dilluns 23 de juny:
	\begin{itemize}
		\item \textbf{Usuaris administradors:} hem inclòs a la secció de bootstrapping (al fitxer \texttt{Bootstrap.java}
		a l'arrel del directori \texttt{src/main/java}) de l'aplicació un usuari administrador,
		amb el qual es pot fer login (les credencials es troben al mateix fitxer de bootstrapping) i es pot comprovar
		com, si intenta jugar una partida, el sistema l'informa que no és un jugador.
		\item \textbf{Partides amb penalització:} les partides amb penalització comencen ara amb la puntuació inicial
		correcta.
	\end{itemize}
	
\end{document}